% example.tex
\documentclass{article}
\usepackage{ctex}    % 中文支持
\usepackage{cgmathnotation}
\usepackage{geometry}

\geometry{a4paper, margin=2.5cm} % 页面设置

\title{数学符号使用示例}
\author{示例文档}
\date{\today}

\begin{document}
\maketitle

\section{基础数学符号使用}

\subsection{矩阵和向量}

基本的矩阵和向量表示:
\[
    \mat{A}, \mat{B}, \mat{C} \quad \text{(矩阵)}
\]
\[
    \vect{x}, \vect{y}, \vect{z} \quad \text{(向量)}
\]

向量分量:
\[
    \vect{v} = (\veccomp{v}{x}, \veccomp{v}{y}, \veccomp{v}{z})
\]

\subsection{微分算子}

一阶偏导数:
\[
    \pder{f}{x}, \quad \pder{f}{y}, \quad \pder{f}{z}
\]

高阶偏导数:
\[
    \pder[2]{f}{x}, \quad \pder[3]{f}{x}, \quad \pder[n]{f}{x}
\]

混合偏导数:
\[
    \pmder{f}{x}{y} = \frac{\partial^2 f}{\partial x\partial y}
\]

全微分:
\[
    \tder{y}{x}, \quad \tder[2]{y}{x}, \quad \tder[n]{y}{x}
\]

\subsection{物理量和单位}

物理量表示:
\[
    \phys{F} = \scalar{m}\vect{a} = \siunit{9.81}{N}
\]

温度变化:
\[
    \Delta T = \siunit{273.15}{K} = \siunit{0}{\celsius}
\]

\subsection{向量运算}

点积和叉积:
\[
    \dotprod{\vect{a}}{\vect{b}} = |\vect{a}||\vect{b}|\cos\theta
\]
\[
    \crossprod{\vect{a}}{\vect{b}} = |\vect{a}||\vect{b}|\sin\theta\,\vect{n}
\]

向量范数:
\[
    \norm{\vect{x}} = \sqrt{\dotprod{\vect{x}}{\vect{x}}}
\]

\subsection{矩阵运算}

基本运算:
\[
    \mat{C} = \mat{A}\mat{B}
\]

转置和逆:
\[
    \mat{D} = \trans{\mat{A}}\inv{\mat{B}}
\]

矩阵元素:
\[
    \matelem{A}{i}{j} = a_{ij}
\]

\subsection{概率统计}

概率:
\[
    \prob{X > x} = 1 - F(x)
\]

期望和方差:
\[
    \expect{X} = \mu, \quad \var{X} = \sigma^2
\]

\section{应用示例}

\subsection{热传导方程}
一维热传导方程:
\[
    \pder{\phys{T}}{t} = \alpha \pder[2]{\phys{T}}{x}
\]

\subsection{运动方程}
牛顿第二定律:
\[
    \phys{F} = \scalar{m}\tder[2]{\vect{x}}{t}
\]

\subsection{电磁场方程}
Maxwell方程组(一部分):
\[
    \operator{\nabla} \times \vect{E} = -\pder{\vect{B}}{t}
\]

\end{document}